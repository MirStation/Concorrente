\documentclass[a4paper,11pt]{article} % Formato do papel, tipo de documento e tamanho da fonte.
\usepackage[utf8]{inputenc}
\usepackage[brazil]{babel} % Hifenização em português
\usepackage[T1]{fontenc} % Caracteres com acentos são considerados como um bloco
\usepackage{ae} % Arruma a fonte quando usa o pacote acima
\usepackage{amssymb} % Caracteres matemáticos especiais
\usepackage[pdftex]{graphicx} % Para inserir figuras    
\usepackage{indentfirst}
\usepackage{float}

\setlength{\parindent}{1cm}
%\renewcommand{\theenumi}{\Alph{enumi}}

\title{
	\vspace{0 mm}
	\huge{\textbf{MAC0438 - Programação Concorrente}} \\
	\vspace{3 mm}
	\huge{EP2 - Cálculo do número de Euler}
	\vspace{0 mm}
}

\author	{
	\Large{{ Antônio Miranda - Igor Canko Minotto}}	\\
}
\date{\Large{{ 22 de maio de 2014}}}


\usepackage{graphicx}
\begin{document}


\maketitle

\pagebreak 
\tableofcontents

\pagebreak
\setcounter{section}{-1}

\section{Introdução}
  Para calcular o número de Euler, utilizamos a fórmula sugerida: \linebreak 
  \centerline{ $e =\sum\limits_{n=0}^\infty \frac{1}{n!}$ } 
  No nosso programa, utilizamos uma thread produtora para calcular os termos da somatória e outras $m-1$ threads consumidoras, sendo $m$ o primeiro argumento da linha de comando. 
  
\section{Ambiente}
  Configuração da Máquina.

\section{Método}
  Demos um valor de entrada 1e-2000, paras as opções f e m. Para medir o tempo de execução,
fizemos a medição dentro do código utilizando a função clock\_gettime() da librt (Realtime Extensions library).
Pegamos o valor inicial do relógio no começo da função main, e o valor final logo antes do ponto de retorno da função.


  Como o tempo foi medido?
  Quantas repetições para cada um?

\section{Análise dos Resultados}
  Gráficos:
    - Para cada quantidade de threads, média aritmética e desvio padrão.
  Análise: foi o esperado?
\end{document}

